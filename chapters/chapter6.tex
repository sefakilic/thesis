
\chapter{Conclusion}

Controlling the rate of initiation of transcription is the most economical way
to regulate gene expression in Bacteria. Despite more than half a century of
research, there are still many uncharacterized transcription factors even in
the most extensively studied bacterial species, such as \textit{Escherichia
  coli} or \textit{Bacillus subtilis}. Continuing research on bacterial
transcriptional regulation will help to characterize the TFs not only in model
organisms but also in other bacterial species of clinical and agricultural
interest. Furthermore, it is important to investigate the relationship between the
changes in the environmental conditions and the adaptation of the
transcriptional regulation, in order to model how Bacteria respond to
environmental stimuli. The standard tool to study the evolution of
transcriptional regulation is comparative genomics, which has been used
frequently to characterize TFs in several species and to study the evolution of
TFs, their binding motifs and the composition of the regulatory networks they control.

The increasing amount of experimental data creates new opportunities for
comparative genomics to analyze the transcriptional regulation in
Bacteria. However, the need to use and deploy comparative genomics approaches
to study transcriptional regulation of Bacteria also poses questions regarding
the main steps of the comparative genomics pipeline: data collection, transfer
of knowledge from one species to another, and comparative analysis. This
dissertation identifies the problems associated with each of these steps and
describes the research performed to address them under a unified framework.

\section{Problems}

\subsection{The storage and organization of experimental data}

The first step of the comparative genomics pipeline is the collection of
experimentally verified TF binding site data. There are several issues with the
existing databases compiling the published data.

Existing TFBS databases usually mix experimentally validated and
computationally predicted binding sites and do not provide functionality to
separate them clearly. Computational models built using the data in these
databases are hence prone to bias, which could lead to inaccurate comparative
genomics analysis further down the pipeline.

Existing databases are not comprehensive in terms of the data across the entire
Bacteria domain. Most databases are organism-specific and establish their own
standards to define data provenance and classify data, complicating the data
collection for comparative genomics analysis. Other databases that are not
organism-specific have not been updated in several years and miss significant
amounts of the data that has become available in recent years.

The existing data available in these databases are not integrated with the
major biological sequence databases that would make it more accessible. Also,
given that some databases have not been updated recently, there is no certainty
that the data will be accessible in the long term.

\subsection{The methods for motif and regulon transfer}
Following data collection, the next step in the comparative genomics pipeline
is the transfer of available TF binding information from one species to
another. There has been two main approaches: motif-based and network-based
methods. In the former, one assumes the conservation of the TF binding motif,
whereas and the latter is based on the conservation of the regulon composition.

Although motif-based methods have been more commonly used, there are
applications of network-based methods which performed well to reconstruct the
TF regulon in target species. The lack of comprehensive TF binding site data
prevented the systematic assessment of the transfer methods so far, precluding
the principled design of possible transfer methods in the context of
comparative genomics analysis (e.g.\ motif-based transfer among closely-related
species and network-based transfer for distant ones).


\subsection{Comparative genomics}

The final step of the pipeline is to compare predicted binding sites and
regulons across all target species in order to perform evolutionary analyses of
TF regulation.

The scarcity of the binding site data makes its aggregation from multiple
species critical for the accuracy of the analysis. The most important issue
with existing frameworks is the lack of methods to combine binding evidence
from multiple species in a systematic way in order to build a generic motif to
be used for site search on each species. Especially for the analysis of species
having divergent TF-binding motifs, it is essential to include all the
available TF-binding evidence in order to construct TF-binding motifs that
would be efficient at detecting sites in each target genome.

Existing frameworks limit the user to a set of preprocessed genomes available
in their systems. This prevents these frameworks from being used to
characterize transcriptional regulation in species with recently sequenced
genomes.

The likelihood of regulation of a given gene is usually assessed with PSSM
scoring of its promoter region. With the raw PSSM score, however, it is
difficult to assess the confidence level of the detected site. A threshold has
to be defined in most cases, which could lead to too many false negatives or
false positives.

None of the existing comparative genomics platforms perform formal ancestral
state reconstruction to infer if the regulation is likely to have existed in
shared ancestors of target species. The lack of formal methods may lead to
inaccurate and not reproducible conclusions about the evolution of TF
regulation in species of interest.

\section{Results}

\subsection{Development of a TFBS database}

To capture the TF-DNA interaction data across all Bacteria domain, CollecTF, a
database of experimentally validated binding sites was built
(Chapter~\ref{chap:collectf}). To ensure the data quality, CollecTF adopts a
thorough process for the curation of publications that report evidence for TF
binding and regulation of gene expression. In addition to in-house efforts for
collecting data, it allows first-hand data submission from the experimental
researchers. It provides fully customizable access to the stored data which can
be browsed by TF and TF-family, taxonomy and experimental support. The
available data is integrated with major biological sequence databases NCBI
RefSeq and EMBL-RBI UniProtKB\-. CollecTF also generates GO annotations to
formalize the relationship between a TF and the chromosomal DNA it binds and
the genes upon which such binding has a transcriptional regulatory
effect. Finally, the supporting experimental evidence is mapped to standardized
Evidence Ontology terms.

\subsection{Assessment of methods for motif and regulon transfer}

The evaluation of motif transfer metrics showed that precision-recall (PR)
area-under-the-curve (AUC) is a better measurement than widely-used receiver
operating characteristic (ROC) AUC\@
(Figure~\ref{fig:transfer-evaluation-comparison}). The available transfer methods
were benchmarked using the collected experimental binding site data. The
results showed that motif-based methods that assume the conservation of the TF
binding motif are consistently better at reconstructing the motif and regulon
in target species than the network-based methods which assume the conservation
of regulon composition (Figure~\ref{fig:comparison-transfer-methods}). It was also
found that the efficiency of the motif-based methods decreases sharply with
increasing phylogenetic distance.

\subsection{Development of a comparative genomics framework}

Leveraging the collected data and the benchmark of existing transfer methods,
CGB, a comparative genomics framework for bacterial transcriptional regulation
analysis was built (Chapter~\ref{chap:cgb}). Unlike similar existing systems,
it does not restrict the set of genomes that can be analyzed; any genome
available at NCBI RefSeq database can be used for the analysis. In addition,
CGB adopts a Bayesian approach to infer regulons in target genomes. Rather than
reporting PSSM scores only, CGB computes posterior probability of regulation
for each operon (Figure~\ref{fig:pipeline}). It constructs orthologous groups
of genes and performs ancestral state reconstruction for each orthologous
group. In addition to results reported in a tabular format, the results are
visualized as a heatmap where rows and columns correspond to orthologous groups
and species, respectively. A given cell of the heatmap is colored based on the
regulation state (regulation, not regulation or absence) of the corresponding
gene. CGB is publicly available as a Python library and stand-alone tool with a
graphical interface.

The platform was tested by replicating a previously published analysis of LexA
regulation in the Firmicutes and the Actinobacteria. The analysis demonstrated
the importance of combining TF binding motif from multiple species for better
detection of binding sites in all target genomes
(Figure~\ref{fig:heatmap_1}). The results are consistent with the original
analysis that was performed using other available tools
(Figure~\ref{fig:heatmap_2}). Furthermore, CGB was used to characterize and
perform a comparative analysis of the Verrucomicrobia LexA
(Chapter~\ref{chap:verrucomicrobia}). The analysis reported significant
differences in the size and composition of the LexA regulatory network of the
Verrucomicrobia regulatory network.

\section{Significance of the Research}

The research described in this dissertation aims to accelerate the comparative
genomics research on transcriptional regulation in Bacteria by implementing
solutions to the problems identified in each major step of the pipeline. The
efforts for collecting the experimental data and developing an automated
framework would obviate the compilation of data and the development of custom
scripts for the analysis, therefore, help the researcher to focus on the
most important step which is the interpretation of the results.

\subsection{Development of a TFBS database}

The collection of the experimental TFBS data in a central and transparent
repository and its customizable access ease the data collection step. Instead
of manually compiling data from publications or combining data from different
organism-specific databases, CollecTF proves to be useful as a comprehensive
resource by serving the preprocessed binding site data and their genomic
context with export options.

Using experimentally validated binding site data exclusively ensures that
models fit using the data are not biased. The mapping of used experimental
techniques to standard Evidence Ontology terms formalizes the experimental
procedure followed to detect each binding site.

The integration with major sequence databases makes the data in CollecTF widely
accessible and safe in the long term. Also, its visibility incentivizes the
authors for direct submission of their data to CollecTF.

\subsection{Assessment of methods for motif and regulon transfer}

The benchmark of existing methods for motif and regulon transfer is the first
ever large-scale effort to systematically measure their performances under
different phylogenetic proximities of reference and target species.

The results showing that PR AUC is the best measure for transfer accuracy
challenges the common practice of using ROC AUC\@. This result suggesting the
use of PR over ROC in binding site classification problem is also supported by
other studies suggesting the same for other classification problems on
imbalanced datasets.

The benchmark results showing better performance of motif-based methods over
network-based methods are instructive for designing transfer step of the
comparative genomics pipeline which can dramatically affect the final results.

\subsection{Development of a comparative genomics framework}

The ability to analyze any sequenced genome makes possible the characterization
of transcriptional regulatory network of any strain, which might be useful for
research groups that work with particular strains of interest.

Combining evidence from multiple sources by taking the account the phylogenetic
relationship between species allows the comparative analysis of
phylogenetically distant species that might have different binding motifs.

Determining a PSSM score threshold is a difficult task since it depends on the
shape of the motif and the genomic content. The Bayesian framework addresses
this issue by computing the posterior probability of regulation for a given
gene. Unlike PSSM scores, the posterior probability is a natural concept with
fixed range, providing easier interpretation of the likelihood of the
regulation. In addition, the Bayesian framework handles tandem TF binding sites
which might be weak to be detected individually by the PSSM search but
functional due to cooperative TF binding. Finally, the Bayesian approach allows
integrating a prior belief of TF binding with the likelihood of observing
putative sites in the promoter.

CGB performs ancestral state reconstruction for each orthologous group of a
gene, inferring the probability of presence and TF-regulation of the gene in
shared ancestors of target species. Ancestral state reconstruction provides a
formal framework to perform evolutionary analysis of TF regulation of the gene.

The availability of CGB as a Python library and a stand-alone tool with
graphical interface ensures the replicability of an analysis performed using
it.

\section{Limitations and Recommendations for Future Research}

\subsection{Development of a TFBS database}

Although CollecTF is a hybrid effort of in-house and direct curations, relying
on in-house curations is unsustainable for the long-term continuity of data
submission.

The strict curation procedure for data quality purposes makes the data
submission a lengthy procedure. In order to make the process easier for the
curator, the extraction of the data in publications can be automated through
text mining. Unlike early publications reporting TF binding site data, many
journals now make the data mining of their publications easy for text
mining. The structured data and the advanced natural language processing
techniques can be utilized to extract the relevant information (e.g.\ genome
and protein accession numbers, binding sites, experimental summary) which would
speed up the curation process. The curator would just have to verify the
extracted data, rather than manually copy it from the publications to CollecTF
data submission forms.

An application programming interface (API) for CollecTF would make it
accessible programmatically for computational tools using TF binding site
data. It would allow the integration of CollecTF to CGB which would enable
direct access of CollecTF data from the comparative genomics tool, similar to
RegPredict-RegTransBase and Visual Footprint-PRODORIC integrations.

\subsection{Assessment of methods for motif and regulon transfer}

Until the study described in this dissertation, the motif- and regulon-transfer
methods had never been systematically evaluated, partially due to scarcity and
distributed nature of the binding site data. CollecTF partially addressed this
issue and made possible the systematic assessment of the methods. The available
data from CollecTF and other databases, however, was dominated by
transcriptional regulators LexA and Fur. The evaluation performed in the study
can be repeated for other TFs as enough amount of their binding site data
becomes available.

The standard model to approximate specific TF-DNA binding is PSSM, which was
shown to be a good approximation of the interaction and has been preferred over
more complex biophysical models and Hidden Markov models. As the
high-throughput data become increasingly available, complex models that require
more data can be investigated, and their performance on transferring available
information can be conducted in a reliable way.

\subsection{Development of a comparative genomics framework}

The integration of CGB with CollecTF would allow using any combination of
binding motifs in CollecTF\@. Thanks to CollecTF's customizable search feature,
binding sites from multiple species that are verified with particular
experimental methods can be searched, aligned and extended if necessary, and
included in the analysis directly.

The EggNOG identifiers can be automatically retrieved and directly assigned to
each orthologous group in order to enrich their functional annotation. Each
EggNOG group contains the functional description, categories, associated GO
annotations, and domain information that would be useful to describe the
elements of the regulatory network in a given species.

In addition to EggNOG identifiers, the NCBI PubMed publication database can be
searched for publications of any relevance to a given gene. The publications
containing relevant information about a given gene could potentially provide
additional context on its regulation by the TF or its lack thereof.

In some cases, there is simply no experimental binding site data to build a TF
binding motif, therefore, no reference motif to start the transfer step to
detect putative sites in target genomes. In such cases, the only alternative is
the motif discovery on the promoters of genes that are known to be
co-regulated. Assuming that functionally important upstream regions such as
binding sites should be more conserved than other non-functional regions, the
overly represented pattern in upstream regions can be determined as the binding
motif which can be used in further steps of the pipeline. Including a motif
discovery to CGB would eliminate the dependency to third-party motif discovery
tools to identify the binding motif to start the analysis.
