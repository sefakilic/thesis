
\documentclass[12pt]{article}
\usepackage[utf8]{inputenc}
\usepackage[T1]{fontenc}
\usepackage{subcaption}
\usepackage{graphicx}
\usepackage{fullpage}
\usepackage{lineno}
\usepackage[numbers,sort&compress]{natbib}
\usepackage{setspace}
\usepackage{booktabs}
\usepackage{array}
\usepackage{multirow}
\usepackage[hidelinks]{hyperref}
\usepackage{wasysym}
\usepackage{mathptmx}  % Times New Roman font

\def\labelitemi{--}
\DeclareMathAlphabet{\mathcal}{OMS}{cmsy}{m}{n} % Reset mathcal font to default

% \chapter{xxx}

\graphicspath {{../}}

\title{Conclusion}
\date{}

\begin{document}
\linenumbers
\doublespacing

\maketitle

Controlling the transcription initiation is the most economical way of gene
regulation in Bacteria. Despite more than half a century of research, there are
still many uncharacterized transcription factors even in most widely-studied
bacterial species such as \textit{Escherichia coli} or \textit{Bacillus
  subtilis}. Continuing research on bacterial transcriptional regulation will
help to characterize the TFs not only in widely-studied model organisms but
also others of clinical and agricultural importance. Also, it is important to
investigate the relationship between the changes in the environmental
conditions and the adaptation of the transcriptional regulation. The standard
tool to study the evolution of transcriptional regulation is comparative
genomics which has been used numerously to characterize TFs in several species
and to study the evolution of TFs, their binding motifs and the regulon
composition.

The increasing amount of experimental data creates new opportunities for
comparative genomics to analyze the transcriptional regulation in
Bacteria. However, it also challenges steps of the comparative genomics
pipeline: the data collection, transfer of knowledge from one species to
another, and comparative analysis. This dissertation identifies the problems
associated with each step and describes the research that attempts to solve
them.

\section{Problems}

\subsection{The storage and organization of experimental data}

The first step of the comparative genomics pipeline is the collection of
experimentally verified TF binding site data. There are several issues with the
existing databases compiling the published data.

\begin{itemize}
\item Existing TFBS databases usually mix the experimentally validated and
  computationally predicted binding sites and do not provide functionality to
  separate them clearly. The computational models built using the data in these
  databases are prone to bias which could lead to inaccurate comparative
  genomics analysis further down the pipeline.

\item Existing databases are not comprehensive in terms of the data across the
  entire Bacteria domain. Most databases are organism-specific and establish
  their own standards and relying on them makes the data collection complicated
  for comparative genomics analysis. Other databases that are not
  organism-specific fail to provide up to date experimental data for Bacteria;
  they have not been updated in several years and miss significant amount of
  the data that has become available in recent years.

\item The existing data available in these databases are not integrated with
  the major biological sequence databases which could make it more
  accessible. Also, given that some databases have not been updated recently,
  there is no certainty that the data will be accessible in the long term.
\end{itemize}

\subsection{The methods for motif and regulon transfer}
Following data collection is to transfer available TF binding information from
one species to another. There has been two main approaches: motif-based and
network-based methods where the former assumes the conservation of the TF
binding motif and the latter is based on the conservation of the regulon
composition.

\begin{itemize}
\item Although motif-based methods have been more commonly used, there are
  applications of network-based methods which performed well to reconstruct the
  TF regulon in target species. The lack of comprehensive TF binding site data
  prevented the systematic assessment of the transfer methods so far.
\end{itemize}

\subsection{Comparative genomics}

The final step of the pipeline is to compare predicted binding sites and
regulons across all target species for evolutionary analysis of the TF regulation.

\begin{itemize}
\item The scarcity of the binding site data makes its aggregation from multiple
  species critical for the accuracy of the analysis. The most important issue
  with existing frameworks is the lack of ability to combine binding evidence
  from multiple species in a systematic way in order to build a generic motif
  to be used for site search. Especially for the analysis of species having
  possibly different TF binding motifs, it is essential to include different
  motifs that would be able to detect sites in each target genome.

\item Existing frameworks limit the user to a set of preprocessed genomes
  available in their systems. The lack of ability to analyze any genome of
  interest prevents these frameworks from being used to characterize
  transcriptional regulation in species with recently sequenced genomes.

\item The likelihood of regulation of a given gene is usually tested with PSSM
  scoring of its promoter region. The raw PSSM score, however, is difficult to
  assess the confidence level of the detected site. A threshold has to be
  defined in most cases which could lead to too many false negatives or false
  positives.

\item None of the existing comparative genomics platforms perform formal
  ancestral state reconstruction to infer if the regulation is likely to have
  existed in shared ancestors of target species. The lack of formal methods may
  lead to inaccurate and not reproducible conclusions about the evolution of TF
  regulation in species of interest.
\end{itemize}

\section{Results}

\subsection{Development of a TFBS database}

To capture the TF-DNA interaction data across all Bacteria domain, CollecTF, a
database of experimentally validated binding sites was built. To ensure the
data quality, CollecTF adopts a thorough process for the curation of
publications that report evidence for TF binding and regulation of gene
expression. In addition to in-house efforts for collecting data, it allows
first-hand data submission from the experimental researchers. It provides fully
customizable access to the stored data which can be browsed by TF and
TF-family, taxonomy and experimental support. The available data is integrated
with major biological sequence databases NCBI RefSeq and EMBL-EBI
UniProtKB\-. CollecTF also generates GO annotations to formalize the relationship
between a TF and the chromosomal DNA it binds and the genes upon which such
binding has a transcriptional regulatory effect. Finally, the supporting experimental
evidence is mapped to standardized Evidence Ontology terms.

\subsection{Assessment of methods for motif and regulon transfer}

The evaluation of motif transfer metrics showed that precision-recall (PR)
area-under-the-curve (AUC) is a better measurement than widely-used receiver
operating characteristic (ROC) AUC\@. The available transfer methods were
benchmarked using the collected experimental binding site data. The results
showed that motif-based methods that assume the conservation of the TF binding
motif are consistently better at reconstructing the motif and regulon in target
species than the network-based methods which assume the conservation of regulon
composition. It was also found that the efficiency of the motif-based methods
decrease sharply with the increasing phylogenetic distance.

\subsection{Development of a comparative genomics framework}

Leveraging the collected data and the benchmark of existing transfer methods,
CGB, a comparative genomics framework for bacterial transcriptional regulation
analysis was built. Unlike similar existing systems, it does not restrict the
set of genomes that can be analyzed; any genome available at NCBI RefSeq
database can be used for the analysis. In addition, CGB adopts a Bayesian
approach to infer regulons in target genomes. Rather than reporting PSSM scores
only, CGB computes posterior probability of regulation for each operon. It
constructs orthologous groups of genes and performs ancestral state
reconstruction for each orthologous group. In addition to results reported in a
tabular format, the results are visualized as a heatmap where rows and columns
correspond to orthologous groups and species, respectively. A given cell of the
heatmap is colored based on the regulation state (regulation, not regulation or
absence) of the corresponding gene. CGB is publicly available as a Python
library and stand-alone tool with a graphical interface.

The platform was tested by replicating the analysis of LexA regulation in
Firmicutes and Actinobacteria. The analysis demonstrated the importance of
combining TF binding motif from multiple species for better detection of
binding sites in all target genomes. The results are consistent with the
original analysis that was performed using other available tools. Also, the CGB
was used to characterize and perform comparative analysis on Verrucomicrobia
LexA-binding motif. The analysis reported significant differences in the size
and composition of the LexA regulatory network of the Verrucomicrobia.

\section{Significance of the Research}

The research described in this dissertation aims to accelerate the comparative
genomics research on transcriptional regulation in Bacteria by implementing
solutions to the problems identified in each major step of the pipeline. The
efforts for collecting the experimental data and developing an automated
framework would obviate the compilation of data and the development of custom
scripts for the analysis, therefore, help the the researcher to focus on the
most important step which is the interpretation of the results.

\subsection{Development of a TFBS database}

\begin{itemize}
\item The collection of the experimental TFBS data in a central and transparent
  repository and its customizable access ease the data collection step. Instead
  of manually compiling data from publications or combining data from different
  organism-specific databases, CollecTF proves to be useful as a comprehensive
  resource  by serving the preprocessed binding site data and their
  genomic context with export options.

\item Using experimentally validated binding site data exclusively ensures that
  models fit using the data are not biased. The mapping of used experimental
  techniques to standard Evidence Ontology terms formalizes the experimental
  procedure followed to detect each binding site.

\item The integration with major sequence databases makes the data in CollecTF
  widely accessible and safe in the long term. Also, its visibility
  incentivizes the authors for direct submission of their data to CollecTF.
\end{itemize}

\subsection{Assessment of methods for motif and regulon transfer}

\begin{itemize}
\item The benchmark of existing methods for motif and regulon transfer is the
  first ever large-scale effort to systematically measure their performances
  under different phylogenetic proximities of reference and target species.

\item The results showing that PR AUC is the best measure for transfer accuracy
  challenges the common practice of using ROC AUC\@. This result suggesting the
  use of PR over ROC in binding site classification problem is also supported
  by other studies suggesting the same for other classification problems on
  imbalanced datasets.

\item The benchmark results showing better performance of motif-based methods
  over network-based methods are instructive for designing transfer step of the
  comparative genomics pipeline which can dramatically affect the final results.
\end{itemize}

\subsection{Development of a comparative genomics framework}

\begin{itemize}

\item The ability to analyze any sequenced genome makes possible the
  characterization of transcriptional regulatory network of any strain, which
  might be useful for research groups that work with particular strains of
  interest.

\item Combining evidence from multiple sources by taking the account the
  phylogenetic relationship between species allows the comparative analysis of
  phylogenetically distant species that might have different binding motifs.

\item Determining a PSSM score threshold is a difficult task since it depends
  on the shape of the motif and the genomic content. The Bayesian framework
  addresses this issue by computing the posterior probability of regulation for
  a given gene. Unlike PSSM scores, the posterior probability is a natural
  concept with fixed range, providing easier interpretation of the likelihood
  of the regulation. In addition, the Bayesian framework handles tandem TF
  binding sites which might be weak to be detected individually by the PSSM
  search but functional due to cooperative TF binding. Finally, the Bayesian
  approach allows integrating a prior belief of TF binding with the likelihood
  of observing putative sites in the promoter.

\item CGB performs ancestral state reconstruction for each orthologous group of
  a gene, inferring the probability of presence and TF-regulation of the gene
  in shared ancestors of target species. Ancestral state reconstruction
  provides a formal framework to perform evolutionary analysis of TF regulation
  of the gene.

\item The availability of CGB as a Python library and a stand-alone tool with
  graphical interface ensures the replicability of an analysis performed using
  it.
\end{itemize}

\section{Limitations and Recommendations for Future Research}

\subsection{Development of a TFBS database}

\begin{itemize}

\item Although CollecTF is a hybrid effort of in-house and direct curations,
  relying on in-house curations is unsustainable for the long-term continuity
  of data submission. The promotion of database

\item The strict curation procedure for data quality purposes makes the data
  submission a lengthy procedure. In order to make the process easier for the
  curator, the extraction of the data in publications can be automated through
  text mining. Unlike early publications reporting TF binding site data, many
  journals now make the data mining of their publications easy for text
  mining. The structured data and the advanced natural language processing
  techniques can be utilized to extract the relevant information (e.g.\ genome
  and protein accession numbers, binding sites, experimental summary) which
  would speed up the curation process. The curator would just have to verify
  the extracted data, rather than manually copy it from the publications to
  CollecTF data submission forms.

\item An application programming interface (API) for CollecTF would make it
  accessible programmatically for computational tools using TF binding site
  data. It would allow the integration of CollecTF to CGB which would enable
  direct access of CollecTF data from the comparative genomics tool, similar to
  RegPredict-RegTransBase and Visual Footprint-PRODORIC integrations.
\end{itemize}

\subsection{Assessment of methods for motif and regulon transfer}

\begin{itemize}

\item Until the study described in this dissertation, the motif- and
  regulon-transfer methods had never been systematically evaluated, partially
  due to scarcity and distributed nature of the binding site data. CollecTF
  partially addressed this issue and made possible the systematic assessment of
  the methods. The available data from CollecTF and other databases, however,
  was dominated by transcriptional regulators LexA and Fur. The evaluation
  performed in the study can be repeated for other TFs as enough amount of
  their binding site data becomes available.

\item The standard model to approximate specific TF-DNA binding is PSSM, which
  was shown to be a good approximation of the interaction and has been
  preferred over more complex biophysical models and Hidden Markov models. As
  the high-throughput data become increasingly available, complex models that
  require more data can be investigated, and their performance on transferring
  available information can be conducted in a reliable way.
\end{itemize}

\subsection{Development of a comparative genomics framework}

\begin{itemize}
\item The integration of CGB with CollecTF would allow using any combination of
  binding motifs in CollecTF\@. Thanks to CollecTF's customizable search feature,
  binding sites from multiple species that are verified with particular
  experimental methods can be searched, aligned and extended if necessary, and
  included in the analysis directly.

\item The EggNOG identifiers can be automatically retrieved and directly
  assigned to each orthologous group in order to enrich their functional
  annotation. Each EggNOG group contains the functional description,
  categories, associated GO annotations, and domain information that would be
  useful to describe the elements of the regulatory network in a given species.

\item In addition to EggNOG identifiers, the NCBI PubMed publication database
  can be searched for publications of any relevance to a given gene. The
  publications containing relevant information about a given gene could
  potentially provide additional context on its regulation by the TF or its
  lack thereof.

\item In some cases, there is simply no experimental binding site data to build
  a TF binding motif, therefore, no reference motif to start the transfer step
  to detect putative sites in target genomes. In such cases, the only
  alternative is the motif discovery on the promoters of genes that are known
  to be co-regulated. Assuming that functionally important upstream regions
  such as binding sites should be more conserved than other non-functional
  regions, the overly represented pattern in upstream regions can be determined
  as the binding motif which can be used in further steps of the
  pipeline. Including a motif discovery to CGB would eliminate the dependency
  to third-party motif discovery tools to identify the binding motif to start
  the analysis.

\end{itemize}



\end{document}