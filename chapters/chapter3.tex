% Assessment of transfer methods

\chapter{Assessment of transfer methods for comparative genomics of regulatory
networks in Bacteria }

\section{Background}

The availability of complete genome sequences for related organisms can be
effectively leveraged to study transcriptional regulatory networks (TRN)
[REFXXX]. In the past decade, comparative genomics approaches have been
routinely employed to study bacterial transcriptional regulatory networks, or
regulons, controlled by a single transcription factor (TF). These studies have
enabled the identification of core network elements and niche-specific
adaptations, providing insights into the evolution of these systems
[REFXXX]. A conventional TRN comparative genomics analysis typically involves
three well-defined steps [REFXXX]. The first step consists in the transfer of
available information on the regulatory network (i.e. known TF-binding sites
and/or regulated genes) to the species under analysis, in order to infer the
TF-binding motif in these target species. The second step involves a
genome-wide search for putative TF-binding sites in the target genomes using
the inferred TF-binding motifs. In the third step, search results from multiple
genomes are integrated across orthologs, based on the assumption that only
orthologs of regulated genes will systematically display TF-binding sites in
their promoter regions.

The power of comparative genomics arises from the aggregation of predictions in
multiple genomes under the assumption of functional selection, which
dramatically reduces the number of false positives [REFXXX]. However, the
effectiveness of this approach depends crucially on the success of the initial
transfer step. Information from a reference TRN can be transferred through the
assumption of a conserved TF-binding motif, a conserved regulon, or a
combination of the two [REFXXX]. As a consequence, several methods have been
proposed to transfer regulatory information across species. The simplest
approach, here called “direct transfer”, is to use the reference TF-binding
motif to search for sites in the target genome. The target motif is hence
implicitly defined as a subset of the highest scoring sites in that genome
[REFXXX]. In “direct discovery”, the direct transfer scheme is further
elaborated by applying a motif discovery or optimization algorithm on a set of
high-scoring sites from the target genome [REFXXX]. The alternative to
TF-binding motif-based transfer is to assume conservation of gene content in
the TRN. Regulon or “network transfer” is typically implemented through the
detection of orthologs for genes in known regulated operons. The promoter
regions of the corresponding operons in target genomes are then analyzed with a
motif discovery tool to elicit the TF-binding motif [REFXXX]. If network
information is not available, a minimal network consisting only of the
TF-coding gene can be postulated under the assumption of self-regulation, and
the TF-binding motif can be inferred with motif discovery tools applied to the
promoter region of the TF-coding gene [REFXXX]. Lastly, motif- and
network-based transfer approaches can be combined in “mixed transfer” to
minimize false positives, at the expense of lowered sensitivity [REFXXX].

The limited availability of experimental data on TF-binding sites has hindered
attempts at systematically assessing and comparing methods for transfer of
transcriptional regulatory networks. Early studies on TRN transfer indicated
that the efficiency of methods based on TF-binding motif transfer faced a sharp
drop-off with increasing sequence divergence among the TF orthologs
[REFXXX]. Later studies exposed the limitations of network-based transfer
methods, due to shortcomings in orthology detection methods and the flexible
nature of bacterial TRNs [REFXXX]. It has been suggested that mixed transfer
methods provide better results [REFXXX], but the scarcity and distributed
nature of TF-binding site data have to date prevented systematic benchmarking
of these methods. In this work, we report the mining and integration of
experimental TF-binding site data from multiple databases into a unified
catalog. Leveraging this resource, we performed a systematic evaluation of TRN
transfer methods across pairs of species and using multiple metrics. In
agreement with previous reports, our results reveal that motif-based transfer
methods perform best, but decay sharply at high TF sequence divergence. In
contrast, the efficiency of network-based transfer methods is poor and weakly
dependent of phylogenetic distance, while mixed methods do not significantly
improve the results of motif-based transfer methods. Our analysis also
highlights the inadequacy of receiver-operating-characteristic (ROC) curves in
heavily unbalanced settings and indicates that the precision-recall (PR)
area-under-the-curve (AUC) is the most informative statistic for assessment of
transfer results. We evaluate predictive measures for transfer accuracy and
discuss their applicability in the context of comparative genomics analysis.


\section{Results and Discussion}

\subsection{Data compilation and evaluation of metrics for the assessment of
  transfer methods}

To perform a systematic analysis of methods for the transfer of transcriptional
regulatory networks in Bacteria, we compiled data from five major databases
reporting experimentally-validated TF-binding sites across the Bacteria
domain. After consolidating replicates, we obtained a catalog of 7,603
TF-binding sites for 344 TFs in 166 species (TABLEXXX). To analyze TRN
transfer, we focused on TF/species pairs that contained at least 10 binding
sites for the same TF in both species. The resulting dataset contains 179
TF-specific species pairs for 15 different TFs across 35 bacterial species and
is dominated by instances of the global transcriptional regulators LexA and Fur
(TABLEXXX).

The establishment of an adequate metric is a necessary and crucial element in a
benchmark study. When transferring TRNs from a reference species to a target
species for comparative genomics, the result of the transfer process is an
inferred TF-binding motif in the target species. Given the inferred and known
TF-binding motifs in the target genome, one can evaluate the accuracy of the
transfer process by directly comparing the motifs or by assessing the
efficiency of the inferred motif at retrieving the known TF-binding sites in a
genome search. Here we focused on the Euclidean distance and the
Kullback–Leibler (KL) divergence as well-established motif comparison functions
based on the position-specific frequency matrix (PSFM) defined by the motif
[REFXXX], and on two standard metrics for classification accuracy based on the
area-under-the-curve (AUC) derived from a TF-binding site search process: the
receiver-operating-characteristic (ROC) AUC and the precision-recall (PR) AUC
[REFXXX]. To assess the efficacy of each method at discriminating the
effectiveness of TRN transfer, we simulated transfers by defining inferred
motifs as noisy pseudo-replicates or permutations of the known collection of
binding sites in the target genomes (FIGUREXXX). We then assessed the quality
of these simulated transfers against the known target motif using the four
metrics outlined above (FIGUREXXX).

As it can be seen in FIGUREXXX, both the Euclidean distance and KL-divergence
perform only moderately well at discriminating the results of simulated noisy
transfers (containing 50\% and 75\% random sites) from completely random or
permuted motifs. This result is partly because the expectation for random
motifs is not to yield maximum distance, narrowing the useful range of motif
comparison metrics. The two other contributing factors are the
high-dimensionality of TF-binding motifs, which is known to decrease the
relative contrast of L-norms [REFXXX], and the presence of low information
bearing positions in most TF-binding motifs. Low information positions are
intrinsically close in PSFM space, artificially increasing the similarity
between motifs for both metrics [REFXXX]. As a result, in both cases, random
and permuted motifs display a considerable spread, leading to significant
overlap with the results obtained for simulated noisy transfers. In practice,
transfer methods frequently generate motifs comparable to the noisy transfers
simulated here, and their overlap with random controls therefore complicates
the interpretation of transfer results.

Accuracy metrics based on a genome-wide search for known TF-binding sites
should in principle provide a more informative metric of the effectiveness of
the transfer process since they evaluate the ability of the inferred motif to
locate true binding sites in the target genome. In contrast with motif
comparison methods, the expectation for accuracy metrics is hence that
incorrect or random transfers should yield very low AUC values. However, this
does not happen for the ROC-AUC, a widely adopted metric in bioinformatics
[REFXXX]. This result is due to the large class imbalance in the TF-binding
search problem, where a handful of known true sites must be distinguished from
the genome background. Even though ROC curves scale properly with class
imbalance [REFXXX], they are ill-suited to discriminate between classifiers in
a heavily imbalanced context, because the negative class dominates the
computation of the ROC-AUC [REFXXX]. The net result of this effect is a
compression of AUC scores for noisy motifs into a very narrow range (0.9-1.0),
making discrimination between near-optimal and noisy transfers almost
impossible. This compression also affects the results obtained for random and
permuted motifs, which spread all the way up to 0.95 AUC scores, further
complicating the interpretation of transfer results. By focusing on the ratio
of true and false positives (precision) and otherwise ignoring the negative
class, the PR-AUC generates scores are not compressed by class imbalance
[REFXXX]. As it can be seen in FIGUREXXX, the PR-AUC effectively exploits its
range to discriminate between noisy transfers and systematically assigns very
small values to random and permuted motifs. Hence, the PR-AUC provides the most
effective metric for the benchmarking of TRN transfer methods and was used in
all subsequent analyses reported here.

\subsection{Comparison of transfer methods}

Motif-based and network-based transfer methods rely on different assumptions
about the evolutionary dynamics of transcriptional regulatory networks. The
former assume that the TF-binding motif is conserved to some extent while the
latter assume that the gene components of the regulon are conserved. As a
result, it is presumed that motif-based methods will perform poorly at large
phylogenetic distances due to expected divergence in the TF-binding motif,
whereas network-based methods are expected to be more resilient to phylogenetic
distance if the biological function of the regulatory network is
preserved. Interestingly, there is evidence supporting and invalidating both
assumptions and their corollaries. The SOS response transcriptional regulator
LexA, for instance, has been shown to target widely diverging motifs in
relatively close species [REFXXX], whereas some transcriptional regulators,
like the heat-shock response repressor HrcA or the arginine repressor ArgR, are
known to preserve their binding motif across Bacteria to different extents
[REFXXX]. On the other hand, regulon composition has been documented to vary
significantly even among closely related species [REFXXX]. Furthermore,
CRP/FNR-type regulators have been shown to control completely different
networks using closely related motifs across Bacteria [REFXXX].

Here we tested the robustness of TRN transfer methods by evaluating the PR-AUC
of inferred TF-binding motifs in 179 TF-specific species pairs, using three
motif-based and three network-based methods, as well as a combination of motif-
and network-based methods (Additional file 4). The motif-based transfer methods
include direct transfer and direct discovery methods. In the direct transfer,
the reference collection of TF-binding sites is used directly to determine the
inferred collection by searching promoter regions in the target genome. In
direct discovery, the results of a relaxed search and their surrounding regions
are used as input for a motif discovery algorithm, with the goal of generating
a motif better adapted to the target genome. The network-based transfer methods
evaluated differ in how they map genes regulated in the reference genome to the
target genome. This mapping can be based on the detection of direct orthology
for genes in regulated operons, their functional assignment using Clusters of
Orthologous Groups (COGs) or orthology detection using their interacting
network partners. The mixed approach combines a relaxed TF-binding motif search
with the restriction that identified sites must be associated with genes mapped
with any of the three network-based transfer approaches.

In agreement with previous research, the results shown in Figure 2 reveal that
the effectiveness of motif-based transfer methods declines rapidly with
decreasing sequence similarity between the TF protein sequences [REFXXX]. In
contrast, the results of network-based transfer methods show only moderate
correlation with protein sequence similarity, but these methods perform poorly
when compared to motif-based transfer methods. Among the three mapping modes
analyzed for network-based transfer, the direct ortholog mode provides the best
results but is still only able to generate successful transfers in 15\% of the
cases. The poor efficiency of network-based transfer methods supports previous
research highlighting the evolutionary flexibility of bacterial transcriptional
regulatory networks, which decreases their expected overlap in gene
composition5, 19–21XX]. The low efficiency of network-based transfers could,
therefore, stem from an inability of these transfer methods to identify
conserved regulated genes (low recall) or from the inclusion of too many
orthologs without conserved regulation in the transfer process (low
precision). The interplay between these factors should explain the significant
differences observed between network transfer modes since these variants are
intended to progressively relax the concept of orthology in order to enhance
recall. To analyze their relative contributions, we computed the Spearman
correlation coefficient between the search PR-AUC reported in Figure 2 and the
precision/recall of the network transfer process for the different transfer
modes. We find that recall (r=0.115), rather than precision (p=0.106), is the
dominant factor for the more restrictive ortholog mode. This indicates that
detecting enough orthologs with conserved regulation is critical for proper
motif inference. However, the situation is reversed for the more relaxed COG
(r=0.089; p =0.191) and interaction (r=-0.108; p =0.140) modes. These results
suggest that the increase in mapped orthologs that are not regulated in the
target genome (loss of precision) overcomes any substantial enhancement in
recall achieved by relaxed mapping modes (Additional file 5).

In contrast with network-based transfer methods, the different implementations
of motif-based transfer yield very similar results (Figure 2). Using the
reference motif to search promoter regions and define the putative target motif
(direct transfer) provides results comparable to those obtained with other
motif-based transfer methods and robust with respect to the specific threshold
used to define the motif (Additional file 6). The use of MEME in direct
discovery transfers to rediscover the TF-binding motif, which has been
postulated to refine and better adapt the inferred motif to the target
genom10XXX], does not provide significant improvements over the direct
transfer. In fact, when performing motif discovery in the promoter region
surrounding the identified sites, MEME may identify other genomic elements
(e.g. Pribnow boxes) as the best motifs, decreasing the accuracy of the
method. Performing motif inference on the identified sites surrounded by random
promoter regions prevents this effect but does not yield a systematic
improvement in PR-AUC values over the direct transfer. Finally, the mixed mode
approach, which has been associated with enhanced specificity [REFXXX], did not
yield a systematic improvement over direct transfer either.

\subsection{Predictive indicators of transfer accuracy}

Our comparative analysis of transfer methods (Figure 2) indicates that even at relatively close phylogenetic distances, both motif- and network-based transfer methods may provide inaccurate results. Hence, manual curation of transfer results, which has been the de facto standard for comparative genomics of TRN in Bacteria [REFXXX], appears to be a necessary requisite to ensure the reliability of any subsequent comparative genomics analyses. Leveraging the TF-binding site catalog compiled here, we attempted to identify predictive indicators of transfer accuracy for motif- and network-based transfer methods. Several studies have exploited sequence similarity in the DNA-binding domain of the TF as a criterion for clustering putative regulatory regions in motif discovery [REFXXX]. The rationale for this approach is that similar DNA-binding domains will target conserved TF-binding motifs. Hence, it is plausible to assume that DNA-binding domain sequence similarity could be an efficient predictor of transfer accuracy for motif-based transfer methods.

To test whether DNA-binding domain sequence similarity is a good predictor of
transfer accuracy, we examined transfer accuracy for two transcription factors
(LexA and Fur) on which we had abundant TF-binding site data and for which the
DNA-binding domain has been experimentally determined [REFXXX]. The results
shown in Figure 3 reveal that DNA-binding domain sequence similarity is not a
universal predictor of transfer accuracy. For LexA, DNA-binding domain sequence
similarity shows a clear correlation (Spearman =0.81) with motif-based transfer
accuracy, but this correlation is completely absent for Fur (=0.01). Our
results, therefore, suggest that for transcription factors (like Fur) targeting
a conserved binding motif, the efficiency of motif-based methods will not
significantly decrease with sequence divergence in the DNA-binding domain. In
contrast, and in agreement with previous findings [REFXXX], the accuracy of
motif-transfer methods is expected to decrease sharply for LexA and other
transcription factors that have significantly altered their binding specificity
through evolution. In this context, DNA-binding domain sequence similarity
provides a more accurate indicator of transfer efficiency than phylogeny
(Additional file 7).

The results that are shown in Figure 3 also indicate that DNA-binding domain
sequence similarity correlates weakly with accuracy for network-based transfer
methods. DNA-binding domain sequence similarity is a proxy for phylogenetic
distance (Additional file 7), and the observed loss of accuracy of
network-based transfer methods is hence congruent with decreased overlap in the
components of regulatory networks for increasing phylogenetic distances
[REFXXX]. It is possible, however, to conceive of other measures that might
function as predictive indices of transfer accuracy for network-based transfer
methods. These methods rely on motif discovery algorithms, like MEME, to infer
the functional motif for the transcription factor in the target species,
providing some theoretical bounds on expected properties of the inferred
motifs. For instance, the information content (IC) of a TF-binding motif is
known to correlate with the number of operons regulated by the TF
[REFXXX]. Hence, if the size of the regulatory network is assumed to remain
relatively constant, we expect the IC of the inferred TF-binding motif to be
similar to that observed in the reference species. In a similar vein, the
distribution of information in a TF-binding motif is related to the structure
of the TF and its mode of binding (e.g. homodimers typically target palindromic
motifs) [REFXXX]. It follows that measures of information distribution in
inferred TF-binding motifs, such as the Gini coefficient [REFXXX], should not
differ much between the reference and inferred motifs under the assumption of
conserved protein structure. We analyzed the predictive power of these indices
on PR-AUC using the complete TF-binding site catalog (Figure 4). While neither
index can reliably identify successful transfers, both indices reveal clear
cutoff values beyond which accurate transfers should not be expected. For both
IC and Gini coefficient, a relative index of 0.5 with respect to the known
reference motif is a strong indicator of unsuccessful transfer (96\% and 94\%
unsuccessful transfers for IC and Gini relative values below 0.5, compared to
83\% for both IC and Gini values above 0.5), and the evidence suggests that
this may also be true for IC and Gini values above 2.

\section{Conclusions}

Transferring known information about transcriptional regulatory networks from
reference to target species is a critical step in comparative genomics
analyses. In this work, we compiled a catalog of known TF-binding sites in
Bacteria and performed a methodic evaluation of assessment metrics in order to
perform the first systematic analysis of different transfer methods. Our
results identify the precision-recall area-under-the-curve as the most reliable
metric for transfer efficiency. We also show that motif-based transfer methods
dramatically outperform network-based approaches, but their effectiveness may
decrease sharply with increasing phylogenetic distance. We evaluate some
predictive indicators of transfer accuracy and show that they are not
consistent or precise enough to enable full automation of TRN transfer. Our
results hence support the long-standing practice of manual curation in
comparative genomics analyses and reveal that the introduction of more
elaborate methods does not clearly benefit motif- or network-based transfer
approaches.

\section{Methods}

\subsection{TF-binding site and genome data}

Experimentally-validated TF-binding sites were compiled from CollecTF [REFXXX],
a Bacteria domain-wide TF-binding site database, and four model-organism
databases: RegulonDB, CoryneRegNet, DBTBS and MtbRegList [REFXXX]. Data from
these databases was downloaded and merged after removal of duplicates and data
without supporting experimental evidence. To evaluate transfer methods across
pairs of species, only regulons with at least 10 experimentally-validated
TF-binding sites for a given TF in both species were used. Complete genome
sequences and annotations for species with available TF-binding site data were
downloaded from the NCBI RefSeq database [REFXXX]. Operon predictions for all
genomes and COG annotations for protein-coding genes were obtained from the
DOOR database [REFXXX].

\subsection{TF-binding site search and motif discovery}

For TF-binding site search, only the regions spanning from -300 bp to +50 bp
relative to the corresponding gene translation start site were considered. Site
search was implemented using custom scripts based on standard Biopython library
functions [REFXXX]. The searched regions were scanned with a sliding window,
evaluating each position with a position-specific scoring matrix (PSSM) based
on a uniform background mononucleotide model [REFXXX]. Motif discovery on
selected sequences was performed with MEME, using command line settings -zoops
-revcomp -dna, and maximum and minimum motif widths, respectively, of 150\% and
50\% of the reference motif width [REFXXX].

\subsection{Transfer methods}

Two main motif-based transfer methods were implemented. In the direct transfer,
a position-specific scoring matrix (PSSM) is built from the reference
collection of binding sites and used to scan the promoter regions of the genome
of interest to identify putative sites [REFXXX]. Under the assumption that
regulon size is conserved to a first approximation, the target motif is defined
as composed of the highest scoring NT sites. NT = ·NR·GT/GR, where NR is the
number of sites in the reference species, GT and GR are genome lengths for
target and reference species, respectively, and is used as a scaling factor to
modulate specificity (=1.25). Direct discovery uses a relaxed scaling factor
(=2.50) to generate a larger collection of putative sites. These sites and
their adjoining intergenic regions are used to rediscover the TF-binding motif
with MEME in direct discovery with true intergenic. For direct discovery with
random intergenic, the genomic intergenic regions are substituted by 100 bp
stretches randomly generated following the intergenic region nucleotide
frequencies.

The network-based transfer was implemented using three different criteria to
map regulated genes in the reference genome to target genomes. In ortholog
mode, orthologs of all genes belonging to regulated operons were detected as
best reciprocal BLAST hits between species pairs using a minimum e-value
threshold of 10-10 [REFXXX]. In COG mode, all genes in the target genome
mapping to the same COG as genes in regulated operons of the reference genome
were considered functional orthologs. In interaction mode, direct interacting
partners for regulated genes in the reference genome were identified using the
STRING database [REFXXX]. The orthologs of these genes on target genomes were
detected through reciprocal BLAST and used to define putative regulatory
networks. In all three cases, the promoters of all target operons containing
mapped genes were then used for motif discovery with MEME.

Mixed transfer uses a relaxed (=2.50) TF-binding site search with the reference
TF-binding motif to define a set of putative sites in the target genome. This
collection of putative sites is filtered by retaining only those sites next to
operons containing genes that have been mapped to the target genome by any of
the network-based transfer methods.

\subsection{Assessment metrics}

Two main types of assessment metrics were used to gauge the efficacy of TF
motif-based transfer methods: motif comparison and performance metrics. Motif
comparison methods involve the direct comparison of the TF-binding motifs
inferred by the transfer method and the motif generated from the known
collection of regulated genes in the target genome. To compare motifs of
different lengths, the two collections of binding sites are shifted with
respect to each other to obtain the gapless alignment that maximizes the
information content (IC) of the joint collection [REFXXX]. The aligned region
of each collection is then used to compute its position-specific frequency
matrix (PSFM). The similarity between the two PSFM is evaluated using either
the Euclidean distance or the Kullback–Leibler (KL) divergence of the inferred
motif from the known target motif [REFXXX]. Performance metrics evaluate the
ability of the inferred motif to retrieve the known target sites when searching
an equal number of promoter regions from the target genome containing, and not
containing, known sites. To assess true positives, the maximum IC alignment
between target and inferred collections is used to compute the offset between
predicted and known sites. The accuracy of the search using the inferred
TF-binding motif is then evaluated as the area-under-the-curve (AUC) of the
receiver-operator-characteristic (ROC) or precision-recall (PR) curve, computed
with the scikit-learn Python library functions [REFXXX].

Control experiments for TRN transfer were generated in two different
ways. Noisy transfers were simulated by defining the inferred motif as a
mixture pseudoreplicate of the known target collection and sequences from the
promoter region of the target genome. Given a mixture weight , the
pseudoreplicate is obtained by sampling, with replacement, (1-)·N sites from
the known target collection and ·N sequences of length L from the promoter
region of the target genome (where N and L are, respectively, the number and
width of sites in the known target collection) (Additional file 3). Noisy
transfers were generated for 0.1, 0.25, 0.5 and 0.75 values of, simulating
increasingly inefficient transfers. Permuted transfers were obtained by
randomly sampling, without replacement, the columns of the known target
motif. A transfer was considered successful if its PR-AUC was larger than two
standard deviations above the mean PR-AUC value observed for permuted
transfers.
