%Abstract Page

\hbox{\ }

\renewcommand{\baselinestretch}{1}
\small \normalsize

\begin{center}
\large{{ABSTRACT}}

\vspace{3em}

\end{center}
\hspace{-.15in}
\begin{tabular}{ll}
Title of dissertation:    & {ENHANCING COMPARATIVE GENOMICS OF}\\
&				      {TRANSCRIPTIONAL REGULATORY NETWORKS} \\
&				      {THROUGH DATA COLLECTION, TRANSFER}\\
&                     {AND INTEGRATION} \\
\ \\
&                          {Sefa Kilic, Doctor of Philosophy, 2016} \\
\ \\
Dissertation directed by: & {Ivan Erill} \\
&  				{Department of Biological Sciences } \\
\end{tabular}

\vspace{3em}

\renewcommand{\baselinestretch}{2}
\large \normalsize

Comparative genomics has proven itself to be an invaluable approach for the
characterization of transcriptional regulatory networks in Bacteria and the
evolutionary analysis of transcriptional regulatory elements: transcription
factors, their binding motifs and regulons they control. The growing influx of
high-throughput experimental data, however, introduces challenges for each step
of the comparative genomics pipeline: the collection of transcription factor
binding site data, the transfer of available information on the regulatory
network to the species under analysis, and the integration of binding site
search results from multiple genomes across orthologs. This dissertation
addresses issues on each step of the workflow and describes a platform for the
analysis of transcriptional regulatory networks in the Bacteria domain. First,
CollecTF, a transcription factor binding site database across the Bacteria
domain, was developed to compile experimentally-validated transcription
factor-binding sites through manual curation. CollecTF provides fully
customizable access to high-quality curated data and integrates it with major
biological resources such as RefSeq, UniProtKB and the Gene Ontology
Consortium. Secondly, different methods for transferring known information from
reference to target species were systematically evaluated for the first time
using a large catalog of known binding sites in Bacteria.  Methods assuming
conservation of the binding were shown to outperform those assuming
conservation of regulon composition. Lastly, a complete comparative genomics
platform (CGB) was built for the analysis of transcriptional regulation on any
annotated bacterial genome. It combines binding evidence from multiple sources
using phylogeny, reports the probability of TF-regulation for each gene through
a Bayesian framework, and performs formal ancestral state reconstruction for
each group of orthologous genes across the species under analysis to
reconstruct the evolutionary history of TF-regulation of the gene. CGB was
benchmarked by replicating a comparative genomics analysis of LexA regulation
in Gram-positive Bacteria, and was later used to characterize LexA regulon in
Verrucomicrobia, a recently established Gram-negative phylum predominant in
many soil bacterial communities.



\par\vfil
